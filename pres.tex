\documentclass[mathserif]{beamer}
\usepackage[utf8]{inputenc}
\usepackage{blindtext}
\usepackage{color}

\usepackage{listings}
\usepackage{fancyvrb}
\usepackage{graphicx}
\usepackage{xcolor,color}
\definecolor{dkgreen}{rgb}{0,0.6,0}
\definecolor{dkpurple}{rgb}{0.4,0.1,0.4}
\definecolor{blue}{rgb}{0.0,0.0,1.0}

\lstset
{
  tabsize=2,
  basicstyle=\small\ttfamily, 
  keywordstyle=\bfseries\color{dkpurple},
  stringstyle=\color{blue}, 
  commentstyle=\color{dkgreen}, 
  showstringspaces=false,
}

\setbeamertemplate{note page}[default]
%\setbeameroption{show notes}
%\setbeameroption{show notes on second screen}

% acceptable: default, Hannover, Goettingen, Szeged, Singapore, Boadilla
\usetheme{Hannover}

% acceptable: default, dove, maybe beaver
\usecolortheme{default}

\title{We are Turing Machines}
  \author[TC CK]{Tyler~I.~Cecil\\Christopher~Koch}
	\institute[New Mexico Tech]{
		Department of Computer Science and Engineering\\
		CSE424 Compiler Writing\\
		New Mexico Tech
	}
	\date{November 10, 2014}
	\subject{Computer Science}

	\begin{document}
		\frame{\titlepage}

    \section{Decisions}
    \begin{frame}
      \frametitle{Entscheidungsproblem}

      % what up, Hilbert?
    \end{frame}

    \section{Turing}
    \begin{frame}
      \frametitle{Turing -- Again}

      We really like him!

      % movie trailer
    \end{frame}

    \section{PDA}
    \begin{frame}
      \frametitle{Push-Down Automata}

      States! Stack! Read-only input!

      Task: Figure out whether a string of binary is of the form 
      \[ 0^n 1^n \] for $n \geq 0$.

      For example, $000111$ and $0011$ should be ``accepted,'' but $00111$ or
      $0101$ should be ``rejected.''

      % Example by us: a^n b^n PDA
    \end{frame}

    % Give 5 minutes for them to try to figure out a^n b^n c^n

    \section{Turing Machines}
    \begin{frame}
      \frametitle{Turing Machines}

      Same thing, but with two stacks! 
      
      Not really, but it's equivalent\ldots

      Church-Turing

      Lambda and G\"odel

      % addition of binary numbers here
    \end{frame}

    \section{Equivalence and Proving It}
    \begin{frame}
      \frametitle{Equivalence and Proof Thereof}

      \begin{center}
      To show something is equivalent to a Turing machine, you only have to
      have it simulate a Turing machine.

      \vspace{10mm}

      \Large You are a Turing Machine!
      \end{center}
    \end{frame}

    \section{Halting Problem}
    \begin{frame}
      \frametitle{Halting Problem}

    \end{frame}

  \end{document}
